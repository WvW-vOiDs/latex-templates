% **************************************************************************** %
%                                theme settings                                %
% **************************************************************************** %

% ============================= CHANGE THESE !!! ============================= %
%%%%% beamer theme (See more: https://deic.uab.cat/~iblanes/beamer_gallery/index_by_theme.html)
\usetheme{AnnArbor}

%%%%% beamer color theme (See more: https://atticus-sullivan.github.io/beamercolortheme/index.html)
\newcommand{\catppuccinstyle}{Mocha}
\newcommand{\catppuccinaccent}{Flamingo}
%%%%% style: Mocha (default), Latte, Frappe, Macchiato
%%%%% accent: Rosewater, Flamingo, Pink, Mauve, Red, Maroon, Peach, Yellow, Green, Teal, Sky, Sapphire, Blue, Lavender... (See more accents: https://catppuccin.com/palette)
%

%%%%% beamr navigation symbols
\setbeamertemplate{navigation symbols}{}

%%%%% math symbols style
\unimathsetup{
    math-style=ISO, % ISO: 大写希腊字母是 italic; TeX(default): 大写希腊字母是 upright.
    % bold-style=ISO, % ISO: 大小写希腊, 拉丁字母粗体都是 italic; TeX(default): 只有小写希腊字母粗体是 italic.
    % sans-style=italic, % upright or italic. Bold sans serif follows from the setting for sans serif; it is completely independent of the setting for bold.
    % nabla=italic, % upright(default) or italic.
    % partial=upright, % upright or italic(default).
    % colon=literal, % The package option colon=literal forces ASCII input ‘:’ to be printed as \mathcolon instead.
    % slash-delimiter % Be sure you know what you are doing if you use this.
}
% ---------------------------------------------------------------------------- %





%%%%% BEAMER:
%%%%% The main colors set in the default color theme are the following:
%%%%% 1. {normal text} is black on white.
%%%%% 2. {alert text} is red on white.
%%%%% 3. {example text} is a dark green (green with 50% black)
%%%%% 4. {structure} is set to a light version of MidnightBlue (more precisely, 20% red, 20% green, and 70% blue).



%%%%% XCOLOR:
%%%%% Within xcolor.sty, the following color names are defined:
%%%%% red, green, blue, cyan, magenta, yellow, black, gray, white, darkgray, lightgray, brown, lime, olive, orange, pink, purple, teal, violet.












% ================================ color theme =============================== %
\usepackage[style=\catppuccinstyle,accent=\catppuccinaccent]{../shared/themes/Catppuccin/beamercolorthemecatppuccin}

% ============================== itemize symbols ============================= %
% \setlist{labelwidth=1em} % Set a positive labelwidth

% 设置 itemize 格式
\setlist[itemize,1]{ % 1级缩进
    label={\small \color{Ctp\catppuccinstyle\catppuccinaccent} \ding{117}}
}
\setlist[itemize,2]{ % 2级缩进
    label={\footnotesize\color{Ctp\catppuccinstyle\catppuccinaccent} \ding{108}}
}
\setlist[itemize,3]{ % 3级缩进
    label={\scriptsize\color{Ctp\catppuccinstyle\catppuccinaccent} \ding{110}}
}

% ============================= enumerate symbols ============================ %

%%%%% 如果你不想要 1.1 1.1.1 这样复杂的格式,可以使用下面的代码
% \setlist[enumerate]{
%     label = {\color{Ctp\catppuccinstyle\catppuccinaccent}\arabic*.},
%     ref = \arabic*
% }

\setlist[enumerate,1]{ % 定义一级列表样式(默认 1., 2., ...)
    label={\color{Ctp\catppuccinstyle\catppuccinaccent}\arabic*.},
    ref=\arabic*,
}
\setlist[enumerate,2]{ % 定义二级列表样式(1.1, 1.2, ...)
    label*={\color{Ctp\catppuccinstyle\catppuccinaccent}\arabic*.},
    ref=\theenumi.\arabic*,
    % leftmargin=2em,
}
\setlist[enumerate,3]{ % 定义三级列表样式(1.1.1, 1.1.2, ...)
    label*={\color{Ctp\catppuccinstyle\catppuccinaccent}\arabic*.},
    ref=\theenumii.\arabic*,
    % leftmargin=4em,
}


% ================================= toc style ================================ %

\setbeamercolor{section in toc}{fg=Ctp\catppuccinstyle\catppuccinaccent}
\setbeamercolor{subsection in toc}{use={structure,normal text},fg=structure.fg!30!normal text.fg}
\setbeamertemplate{section in toc}[sections numbered]
\setbeamertemplate{subsection in toc}[ball]


% ============================= hyper link style ============================= %
\hypersetup{
    % breaklinks=<boolean>, % 允许链接换行
    colorlinks=true, % citations, page references, URLs, local file references, and other links.
    linkcolor=., % Color for normal internal links.
    % anchorcolor=<color>, % Color for anchor text. Ignored by most drivers.
    % citecolor=<color>, % Color for bibliographical citations in text.
    % filecolor=<color>, % Color for URLs which open local files.
    % menucolor=<color>, % Color for Acrobat menu items.
    % runcolor=<color>, % Color for run links (launch annotations).
    urlcolor={cyan!30!normal text.fg!}, % Color for linked URLs.
    % allcolors=<color>, % Set all color options (without border and field options).
}


% =================================== fonts ================================== %
\usefonttheme{professionalfonts}

%%%%% math font

% \usepackage{firamath-otf}
\setmathfont{FiraMath-Regular.otf}[
    Path = ../shared/fonts/FiraMath/
]

% 一些 Fira Math 没有的字符

\setmathfont{Latin Modern Math}[ % TexLive 提供
    range = {bbit,scr,frak,tt,sfup,sfit,bfscr,bffrak,bfsfup,bfsfit}
]
\setmathfont{STIXTwoMath-Regular.otf}[ % TexLive 提供
    range = {cal,bfcal},
    StylisticSet = 1
]
\setmathfont{STIXTwoMath-Regular.otf}[ % TexLive 提供
    range={"213D-"213F} % \symbb{\gamma,\Gamma,\Pi}
]


%%%%% English font
\usepackage{fontspec}
\setmainfont{SarasaTermSCNerd}[ % main font 等价于 roman font
    % Contextuals = Alternate,  % otf 支持
    Path = ../shared/fonts/SarasaTermSCNerd/,
    Extension = .ttf,
    BoldFont = *-Bold,
    ItalicFont = *-Italic,
    BoldItalicFont = *-BoldItalic,
    % SlantedFont = ⟨font name⟩ % 默认使用 ItalicFont
    % BoldSlantedFont = ⟨font name⟩ % 默认使用 BoldItalicFont
    UprightFont = *-Regular,
    % SwashFont = ⟨font name⟩
    % BoldSwashFont = ⟨font name⟩
    % SmallCapsFont = ⟨font name⟩
]
\setsansfont{SarasaTermSCNerd}[
    % Contextuals = Alternate,  % otf 支持
    Path = ../shared/fonts/SarasaTermSCNerd/,
    Extension = .ttf,
    BoldFont = *-Bold,
    ItalicFont = *-Italic,
    BoldItalicFont = *-BoldItalic,
    % SlantedFont = ⟨font name⟩ % 默认使用 ItalicFont
    % BoldSlantedFont = ⟨font name⟩ % 默认使用 BoldItalicFont
    UprightFont = *-Regular,
    % SwashFont = ⟨font name⟩
    % BoldSwashFont = ⟨font name⟩
    % SmallCapsFont = ⟨font name⟩
]

%%%%% Code font
\setmonofont{SarasaTermSCNerd}[
    % Contextuals = Alternate,  % otf 支持
    Path = ../shared/fonts/SarasaTermSCNerd/,
    Extension = .ttf,
    BoldFont = *-Bold,
    ItalicFont = *-Italic,
    BoldItalicFont = *-BoldItalic,
    % SlantedFont = ⟨font name⟩ % 默认使用 ItalicFont
    % BoldSlantedFont = ⟨font name⟩ % 默认使用 BoldItalicFont
    UprightFont = *-Regular,
    % SwashFont = ⟨font name⟩
    % BoldSwashFont = ⟨font name⟩
    % SmallCapsFont = ⟨font name⟩
]

%%%%% CJK font
\usepackage{xeCJK}
\setCJKsansfont{LXGWBright}[
    Path = ../shared/fonts/LXGWBright/,
    Extension = .ttf,
    UprightFont = *-Regular,
    BoldFont = *-Medium,
    ItalicFont = *-Italic,
    BoldItalicFont = *-MediumItalic,
]
\setCJKmonofont{SarasaTermSCNerd}[
    % Contextuals = Alternate,  % otf 支持
    Path = ../shared/fonts/SarasaTermSCNerd/,
    Extension = .ttf,
    BoldFont = *-Bold,
    ItalicFont = *-Italic,
    BoldItalicFont = *-BoldItalic,
    % SlantedFont = ⟨font name⟩ % 默认使用 ItalicFont
    % BoldSlantedFont = ⟨font name⟩ % 默认使用 BoldItalicFont
    UprightFont = *-Regular,
    % SwashFont = ⟨font name⟩
    % BoldSwashFont = ⟨font name⟩
    % SmallCapsFont = ⟨font name⟩
]

% **************************************************************************** %
%                                     other                                    %
% **************************************************************************** %

%%%%% 这些设置是为了使形式为 $\widetilde{G_{+, 0}} \widehat{abc}$ 的公式能够正确显示. (ref: https://tex.stackexchange.com/questions/701542/how-to-get-a-widetilde-with-firamath)
\newfontface{\dvmath}{texgyredejavu-math.otf}[
  NFSSFamily=dvm,
  Script=Math, % <---- important!
]
\DeclareSymbolFont{dvm}{TU}{dvm}{m}{n}
\AtBeginDocument{%
  \renewcommand{\widehat}{\Umathaccent 7 \symdvm "00302\relax}%
  \renewcommand{\widetilde}{\Umathaccent 7 \symdvm "00303\relax}%
}